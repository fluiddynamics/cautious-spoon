\documentclass[12pt]{ltjsarticle}
\usepackage[ipa]{luatexja-preset}
\usepackage{fancybox}
\usepackage{amsmath}
\renewcommand{\refname}{出典}
\addtolength{\oddsidemargin}{-1cm}
\addtolength{\evensidemargin}{-1cm}
\addtolength{\topmargin}{-1cm}
\addtolength{\textwidth}{1.7cm}
\addtolength{\textheight}{1cm}
\begin{document}
\title{\vspace{-5mm}確率統計の問題集}
\maketitle
\section{計算問題}
\begin{enumerate}
\item 次の問いに答えよ
\begin{enumerate}
\item
お金が入っている箱がたくさん並んでいます.
各々の箱の額は不明で,その平均金額を知るために,標本として1箱だけ
無作為に抽出して調べたところ,箱の中には500円が入っていました.
すべての箱の中の平均金額を区間推定してみてください.
ただし,箱の中の金額は分散$30^2$の正規分布に従うとします.
\cite[p.122]{wakui}
\item
お金が入っている中の見えない箱がたくさん並んでいます.
各々の箱の額は不明で,平均金額を知るために,標本として9箱を無作為に
抽出し調べてみました.その結果は次の通りです.\\
\hspace{1cm} 530, 515, 470, 545, 440, 530, 455, 560, 455 \\
この標本から,箱の中の平均金額を信頼度95\%で推定してください.
ただし,箱の中の金額は分散$30^2$の正規分布に従うとします.
\cite[p.127]{wakui}
\item
A県の20歳男子の200人を無作為に抽出したところ,身長の平均は168.0でした.
A研の身長は母分散$6.5^2$の正規分布に従うと仮定できます.
この県の20歳男子全体の平均身長μを信頼度95\%で推定してください.
\cite[p.131]{wakui}
\item
上の例題で,信頼度を95\%ではなく,99\%にした時の信頼区間を求めてください.
\cite[p.131]{wakui}
\item
お金が入っている中の見えない箱がたくさん並んでいます.
各々の箱の額は不明で,平均金額を知るために,標本として
9箱,無作為に抽出し調べた結果は次の通りでした.\\
\hspace{1cm} 530, 515, 470, 545, 440, 530, 455, 560, 455 \\
この標本から,箱の中の平均金額を信頼度95\%で推定してください.
ただし,箱の中の金額は正規分布に従うとします.
\cite[p.132]{wakui}
\item
A県の20歳男子10人を抽出し身長を調べたところ,
その標本の平均身長は168.0,不偏分散は$6.5^2$でした.
この県の20歳男子の平均身長μを信頼度95\%で推定してください.
なお,自由度9の両側5%点は2.26として計算してください.
\cite[p.135]{wakui}
\item
中の見えないたくさんの箱の中にはお金が入っています.
箱の中の金額Xの平均値を知るために,大きさ100の標本を取り出し
調べたところ,次のように標本平均$\overline{X}$と不偏分散$s^2$が求められました.
これらの値から,箱の中の金額の平均値$\mu$を信頼度95\%で推定してください. \\
\hspace{1cm} $\overline{X} = 500, \quad s^2=50^2$\\
\cite[p.142]{wakui}
\item
A県の20歳男子200人を抽出し調べたところ,その標本の平均身長は168.0,
不偏分散は$6.5^2$でした.この県の20歳男子の平均身長$\mu$を信頼度
95\%で推定してください.
\cite[p.142]{wakui}
\item
日本全体のペットの飼育率を調べるために大きさ500の
標本を抽出して標本比率を調べたところ,0.62でした.
これをもとに日本全体のペットの飼育率$R$を信頼度95\%で
推定してください.
\cite[p.145]{wakui}
\item
K工場から出荷されるカップラーメン10個について,その
内容量を調べたところ,次のような結果が得られました.
この標本から,製造されるカップラーメン全体の内容量の分散$\sigma^2$
を信頼度95\%で推定してください.\\
\hspace{1cm} 184.2, 176.4, 168.0, 170.0, 159.1,
177.7, 176.0, 165.3, 164.6, 174.4 \\
\cite[p.147]{wakui}
\item
ある都市の住民の体重の分散$\sigma^2$を推定するために大きさ10の標本を抽出して
調べたところ,不偏分散$s^2$が35.4でした.
この都市の住民の体重の分散$\sigma^2$を信頼度95\%で推定してください.
\cite[p.149]{wakui}
\end{enumerate}
\item
次のデータは正規分布$N\left(\mu, \sigma^2\right)$
からの無作為標本値である.\cite[p.164]{inagaki}
\begin{table}[h]
\begin{center}
\begin{tabular}{rrrrrr}
12.7& 6.6& 5.6& 14.3& 11.4& 10.8 \\
13.8& 11.2& 10.0& 12.8& 7.1& 14.0
\end{tabular}
\end{center}
\end{table}
\begin{enumerate}
\item 分散が$\sigma^2=7$として与えられているとき,$\mu$の
95\%信頼区間を求めよ.
\item 分散$\sigma^2$が未知のとき,$\mu$の95\%信頼区間を求めよ.
\item 分散$\sigma^2$の95\%信頼区間を求めよ.
\end{enumerate}
\item
ある町の有権者300人に候補者Aを支持するかどうか意見を聞いたところ,
180人が支持すると答えた.この調査結果から,候補者Aの支持率$p$の
95\%信頼区間を求めよ.
\cite[p.165]{inagaki}
\item
$n$を自然数とする.
原点$O$から出発して数直線上を$n$回移動する点$A$を考える.
点$A$は,1回ごとに,確率$p$で正の向きに3だけ移動し,
確率$1-p$で負の向きに1だけ移動する.
ここで,$0<p<1$である.
$n$回移動した後の点$A$の座標を$X$とし,$n$回の移動のうち
正の向きの移動の回数を$Y$とする.
\cite{center}
\begin{enumerate}
\item $\displaystyle p=\frac{1}{3}, n=2$の時,
確率変数$X$のとり得る値は,小さい順に
\framebox{ア}, \framebox{イ}, \framebox{ウ}
であり,これらの値をとる確率は,それぞれ
\framebox{エ}, \framebox{オ}, \framebox{カ}
である.
\item $n$回移動したとき,$X$と$Y$の間に$X=\framebox{キ}n+\framebox{ク}Y$
の関係が成り立つ.
確率変数$Y$の平均(期待値)は\framebox{ケ},分散は\framebox{コ}
なので,$X$の平均は\framebox{サ},分散は\framebox{シ}である.
\item $\displaystyle p=\frac{1}{4}$のとき,1200回移動した後の点Aの
座標$X$が120以上になる確率の近似値を求めよう.
(b)により,$Y$の平均は\framebox{ス},標準偏差は\framebox{セ}であり,求める確率は次のようになる.
$$
P\left(X\ge 120\right) =
P\left(\frac{Y-\framebox{ス}}{\framebox{セ}}\ge \framebox{ソ}\right)
$$
いま,標準正規分布に従う確率変数を$Z$とすると,$n=1200$は十分に大きいので,
求める確率の近似値は正規分布表から次のように求められる.
$$
P\left(Z\ge \framebox{ソ}\ \right)=\framebox{タ}
$$
\item
$p$の値がわからないとする.2400回移動した後の点$A$の座標が
$X=1440$のとき,$p$に対する信頼度95\%の信頼区間を求めよう.
$n$回移動したときに$Y$がとる値を$y$とし, $\displaystyle r=\frac{y}{n}$とおくと,
$n$が十分に大きいならば,確率変数$\displaystyle R=\frac{Y}{n}$は近似的に平均$p$,分散
$\displaystyle \frac{p\left(1-p\right)}{n}$の正規分布に従う.
$n=2400$は十分に大きいので,このことを利用し,分散を$\displaystyle \frac{r\left(1-r\right)}{n}$
で置き換えることにより,
求める信頼区間は
\[
\framebox{チ}\le p \le\framebox{ツ}
\]
となる.
\end{enumerate}
\end{enumerate}
\section{数式の問題}
\begin{enumerate}
\item
$\Gamma$関数の定義は以下で与えられます.
\[
\Gamma\left(x\right)=\int_0^\infty t^{x-1}e^{-t}dt,\qquad
x>0
\]
以下のことを示してください.
\begin{eqnarray*}
\Gamma\left(x+1\right)&=&x\Gamma\left(x\right). \\
\Gamma\left(1\right)&=&1. \quad
\Gamma\left(2\right)=1. \\
\Gamma\left(x\right)&=&2\int_0^\infty t^{2x-1}e^{-t^2}dt.
\end{eqnarray*}
\item
$B$関数の定義は以下で与えられます.
\[
B\left(x,y\right)=2\int_0^\frac{\pi}{2}
\cos^{2x-1}\theta\sin^{2y-1}\theta\,d\theta,\qquad
x>0,\ y>0
\]
以下のことを示してください.
\begin{eqnarray*}
B\left(x,y\right)&=&
\frac{\Gamma\left(x\right)\Gamma\left(y\right)}{\Gamma\left(x+y\right)}. \\
B\left(1,1\right)&=&1. \quad
B\left(\frac{1}{2},\frac{1}{2}\right)=\pi. \quad
\Gamma\left(\frac{1}{2}\right)=\sqrt{\pi}. \quad
\Gamma\left(\frac{3}{2}\right)=\frac{\sqrt{\pi}}{2}.
\end{eqnarray*}
\item
正規分布$N\left(\mu,\sigma^2\right)$の確率密度関数は,$k$を定数として
以下のように表せます.
\[
f\left(x\right)=ke^{-\frac{\left(x-\mu\right)^2}{2\sigma^2}}
\]
確率変数$X$は$N\left(\mu,\sigma^2\right)$に従うとします.
以下のことを示してください.
\begin{eqnarray*}
k&=&\frac{1}{\sqrt{2\pi\sigma^2}}. \\
E\left(X\right)&=&\mu. \\
V\left(X\right)&=&\sigma^2.
\end{eqnarray*}
\item
自由度$n$の$\chi^2$分布$\chi_n^2$の確率密度関数は,$k$を定数として
以下のように表せます.
\[
f\left(x\right)=kx^{\frac{n}{2}-1}e^{-\frac{x}{2}},\qquad
x>0
\]
確率変数$X$は$\chi_n$に従うとします.
以下のことを示してください.
\begin{eqnarray*}
k&=&\frac{1}{\Gamma\left(\frac{n}{2}\right)2^{\frac{n}{2}}}. \\
E\left(X\right)&=&n. \\
V\left(X\right)&=&2n.
\end{eqnarray*}
\item
自由度$n$の$T$分布$T_n$の確率密度関数は,$k$を定数として
以下のように表せます.
\[
f\left(x\right)=k\left(1+\frac{x^2}{n}\right)^{-\frac{n+1}{2}}
\]
確率変数$X$は$T_n$に従うとします.
以下のことを示してください.
\begin{eqnarray*}
k&=&\frac{1}{\sqrt{n}B\left(\frac{n}{2},\frac{1}{2}\right)}. \\
E\left(X\right)&=&0.
\end{eqnarray*}
\item
自由度$m,n$のエフ分布$F_n^m$の確率密度関数は,$k$を定数として
以下のように表せます.
\[
f\left(x\right)=kx^{\frac{m}{2}-1}\left(\frac{m}{n}x+1\right)^{-\frac{m+n}{2}}
,\qquad
x>0.
\]
以下の空欄を埋めてください.
\[
k=\fbox{}
\]
\item
ベータ分布$B_E\left(\alpha,\beta\right)$の確率密度関数は,$k$を定数として
以下のように表せます.
\[
f\left(x\right)=kx^{\alpha-1}\left(1-x\right)^{\beta-1},\qquad
0<x<1.
\]
以下の空欄を埋めてください.
\[
k=\fbox{}
\]
\item
確率分布について,次の関係を示してください.
\begin{enumerate}
\item
$X$が$N\left(\mu,\sigma^2\right)$に従うとき,
$aX+b$は$N\left(a\mu+b,a^2\sigma^2\right)$に従う.
ただし$a,b$は定数.
\item
$X$が$N\left(0,1\right)$に従うとき,
$X^2$は$\chi_1$に従う.
\end{enumerate}
\item
$n$は任意の正整数,$k$は$1\le k\le n$なる整数であり,
$p$は$0<p<1$なる実数であるとする.
\begin{enumerate}
\item
等式:
\[
\sum_{r=k}^{n}\binom{n}{r}p^r\left(1-p\right)^{n-r}=
\frac{1}{B\left(k,n-k+1\right)}\int_0^p x^{k-1}\left(1-x\right)^{n-k}dx
\]
が成り立つことを証明せよ.
\cite[p.47]{inagaki}
\item
確率変数$X$がベータ分布$B_E\left(k,n-k+1\right)$
に従うとき,$$ Y=\left(\frac{n+1}{k}-1\right)\left(\frac{1}{1-X}-1\right)$$
は自由度$\left(2k,
2\left(n-k+1\right)\right)$のエフ分布に従うことを示せ.
\cite[p.106]{inagaki}
\item
$X$が二項分布$B_N\left(n,p\right)$に従うとき,等式:
\[
P\left(X \ge k\right)=
P\left(Y\le \left(\frac{n+1}{k}-1\right)\left(\frac{1}{1-p}-1\right)\right)
\]
が成り立つことを示せ.ただし,$Y$は自由度$\left(2k,
2\left(n-k+1\right)\right)$のエフ分布に従う確率変数である.
\cite[p.106]{inagaki}
\item
次の不等式を$p$について解いてください.
\[
Y\le \left(\frac{n+1}{k}-1\right)\left(\frac{1}{1-p}-1\right)
\]
\end{enumerate}
\end{enumerate}
\begin{thebibliography}{9}
\bibitem{wakui} 涌井貞美,意味がわかる統計解析
\bibitem{center} 平成28年度大学入試センター試験 数学Ⅱ・数学B
\bibitem{inagaki} 稲垣宣生,数理統計学
\end{thebibliography}
\end{document}
